
% -------------------------------------------------------------
% PACKAGES
% -------------------------------------------------------------
\usepackage{microtype}                                % Tweak font spacing for aesthetics
\usepackage{cmbright}                                 % Font
\usepackage[utf8]{inputenc}                           % Includes letters with accents
\usepackage[T1]{fontenc}                              % 8-bit encoding with 256 glyphs
\usepackage[colorlinks = true,
            allcolors = black]{hyperref}              % Hyper-references in PDF
\usepackage{subfigure}                                % Multiple figures environment
\usepackage{amsmath}                                  % Standard maths
\usepackage{amssymb}                                  % Standard maths symbols
\usepackage{graphicx}                                 % Include graphics
\usepackage{booktabs}                                 % Professional tables
\usepackage[english]{babel}                           % Babel and language definitions
\usepackage[a4paper,                                  % Paper size
            top = 25mm,                               % Top margin
            bottom = 25mm,                            % Bottom margin
            left = 25mm,                              % Left margin
            right = 25mm]{geometry}                   % Right margin
\usepackage{xcolor}                                   % Advanced colours
\usepackage{soul}                                     % Highlight text and other text things
\usepackage{caption}                                  % Manage captions in floats
\usepackage{textcomp}                                 % Includes euro symbol
\usepackage{ifthen}                                   % Use of conditional \if
\usepackage{fmtcount}                                 % Convert integers to words
\usepackage[per-mode = symbol]{siunitx}	              % To use SI units
\usepackage[version=4]{mhchem}                        % For writing chemical symbols
\usepackage{totcount}                                 % Display points counter at beginning 
\usepackage{xstring}                                  % To manipulate strings
\usepackage{multicol}                                 % Multiple column environment (FS)
\usepackage{lipsum,turnthepage}
\usepackage{wasysym}                                  % Symbols for multiple choice quesitons

% -----------------------------------------------------
% DOCUMENT DEFINITIONS
% -----------------------------------------------------
\pointsinrightmargin                                  % Points on the right margin
\pointformat{\bfseries(\themarginpoints)}             % Points in bold and bracket style
\renewcommand{\partlabel}{\bfseries{(\thepartno)}}    % Bold part numbers
\renewcommand{\subpartlabel}{\bfseries{\thesubpart)}} % Bold part numbers
\pagestyle{headandfoot}                               % Solution header and footer
\header{}{}{}                                         % Solution header
\footer{}{}{\theCourseCode{}                          % Solution footer
            \theCourse{} --- \theExamDate{}
            (\theDiet)} 
\graphicspath{{Images/}}                              % Directory where pictures are stored
\definecolor{myRed}{RGB}{184,0,1}                     % Define new colour
\definecolor{myGrey}{RGB}{211,211,211}                % Define new colour
\checkedchar{\large $\CIRCLE$}
\setlength{\fboxrule}{0pt}

% -----------------------------------------------------
% NEW COMMANDS
% -----------------------------------------------------
\newtotcounter{pointsCounter}                         % Counter for total number of points in parts
\newtotcounter{numberOfSections}                      % Counter for number of sections
\newtotcounter{numberOfQuestions}                     % Counter for number of questions

\newcommand{\sectionStart}{                           % Initiate new sections
    \setcounter{question}{0}                          % (Re)initiate questions counter
    \setcounter{figure}{0}                            % Initiate figure counter
    \setcounter{table}{0}                             % Initiate table counter
    \qformat{\ifthenelse{\equal{\thequestion}{1}}
        {\fbox{\parbox{16cm}{\textbf{\large SECTION \Alph{numberOfSections} \hfill\vrule depth 0.5em width 0pt \\ Question \Alph{numberOfSections}\thequestion}} \hfill\vrule depth 1.5em width 0pt}}
        {\fbox{\parbox{16cm}{\textbf{Question \Alph{numberOfSections}\thequestion}} \hfill\vrule depth 0.8em width 0pt}}
    }
    \stepcounter{numberOfSections}                    % Increment sections counter
}

\newcommand{\callQuestion}[2]{                        % Call new question
    \cleardoublepage \input{#1}                       % Start question in new page
    \addtocounter{pointsCounter}{#2}                  % Add points to counter
    \addtocounter{numberOfQuestions}{1}               % Increment questions counter
    \stepcounter{figure}                              % Increment figure counter
    \stepcounter{table}                               % Increment table counter
}

\newcommand{\callFormulaSheet}[1]{
    \ifthenelse{\equal{\formulaSheet}{Yes} \OR \equal{\formulaSheet}{Y} \OR \equal{\formulaSheet}{yes} \OR \equal{\formulaSheet}{y} \OR \equal{\formulaSheet}{YES}}{\cleardoublepage \input{#1}}{}
}

\newcommand{\headingFormulaSheet}{
    \pagestyle{empty}
    \setlength{\textwidth}{180mm}
    \setlength{\textheight}{265mm}
    \setlength{\topmargin}{-20mm}
    \setlength\oddsidemargin{-12mm}
    \setlength\evensidemargin{-12mm}
    \setlength{\columnsep}{10mm}
    \setlength{\columnseprule}{0.4pt}
    \setlength{\jot}{5pt}
    \setlength\parindent{0pt}
%
    \twocolumn
    \fcolorbox{white}{myGrey}{\begin{minipage}{20em}
        \centering \vspace{4mm} {\Large \textbf{Formula Sheet}} \\ 
        \theCourse\ (\theCourseCode) \\
        {\small \theExamDate{} (\theDiet)} \vspace{3mm} 
    \end{minipage}}
}

% -----------------------------------------------------
% TITLES AND FORMATS
% -----------------------------------------------------
%\qformat{\textbf{\large Question \thequestion} \hfill\vrule depth 1.5em width 0pt} 
                                                      % Format of question title - numbered questions, no parts
\pointname{}                                          % No title for marks
\pointsdroppedatright                                 % Marks at right margin
\framedsolutions                                      % Add frame to solutions
\addto\captionsenglish{                               % Format Figure labels
    \renewcommand\figurename{Figure Q}}
\addto\captionsenglish{                               % Format Table labels
    \renewcommand\tablename{Table Q}}
\makeatletter
\def\fnum@figure{\figurename\Alph{numberOfSections}\thefigure}
\def\fnum@table{\tablename\Alph{numberOfSections}\thetable}
\makeatother
\renewcommand{\solutiontitle}{
   \noindent\textbf{Marking guidelines:}\par\noindent}  % Rename solutions

